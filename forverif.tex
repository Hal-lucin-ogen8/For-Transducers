\documentclass[11pt]{article}


\usepackage{amsmath}
\usepackage{amssymb}
\usepackage{amsthm}
\usepackage{mathtools}
\usepackage{stmaryrd}
\usepackage{tikz}

\usepackage[cleveref,xcolor,hyperref]{knowledge}
\knowledgeconfigure{notion}

\newcommand{\defined}{\coloneqq}
\newcommand{\Nat}{\mathbb{N}}
\newcommand{\Rel}{\mathbb{Z}}
\newcommand{\Bool}{\mathbb{B}}
\newcommand{\set}[1]{\{#1\}}
\newcommand{\setof}[2]{\{#1 \mid #2\}}

\newcommand{\Vars}{\mathbb{V}}

\newcommand{\forCmd}{\kl[\forCmd]{\text{prog}}}
\knowledge{\forCmd}{notion}

\newcommand{\forBool}{\kl[\forBool]{\text{boolexpr}}}
\knowledge{\forBool}{notion}

\newcommand{\forString}{\kl[\forString]{\text{stringexpr}}}
\knowledge{\forString}{notion}

\newcommand{\last}{\mathsf{last}}
\newcommand{\first}{\mathsf{first}}

\newcommand{\For}[3]{\mathsf{for}^{#2} #1. #3}
\newcommand{\If}[3]{\mathsf{if} \, #1 \, \mathsf{then} %
\, #2 \, \mathsf{else} \, #3}
\newcommand{\NoOp}{\mathsf{nop}}
\newcommand{\Seq}[2]{#1 \, ; \, #2}
\newcommand{\Print}[1]{\mathsf{print} \, #1}


\title{Automatic Verification of For Programs}
\author{TODO}


\begin{document}

\maketitle

\begin{abstract}
    In this document we present the theoretical basis for
    the automated verification of polyregular functions.
    These are string to string functions described using
    for-loops. We show how to efficiently translate such functions
    into first-order (resp. monadic second-order) logic formulas
    and how this can effectively be leveraged to verify
    properties of these functions.
\end{abstract}

\section{Introduction}

In this document we present the theoretical basis for
the automated verification of polyregular functions.


\section{Preliminaries}

\subsection{Syntax of Variable Free For Programs}

\begin{figure}[h]
\begin{align*}
    \intro*\forCmd \defined & \,\, \For{x}{d}{\reintro*\forCmd} & 
                        d \in \set{ \leftarrow, \rightarrow },
                        x \in \Vars
                       \\
                &\mid  \If{\forBool}{\reintro*\forCmd}{\reintro*\forCmd} \\
                &\mid  \NoOp \\
                &\mid  \Seq{\reintro*\forCmd}{\reintro*\forCmd} \\
                &\mid  \Print{\forString}
\end{align*}
\caption{Syntax of For Programs}
\end{figure}

\begin{figure}[h]
    \begin{align*}
        \intro*\forBool \defined & \,\,  a(x) & a \in \Sigma, x \in \Vars \\
                            &\mid  \reintro*\forBool \land \reintro*\forBool \\
                            &\mid  \reintro*\forBool \lor \reintro*\forBool \\
                            &\mid  \lnot \reintro*\forBool \\
                            &\mid  x \leq y  & x,y \in \Vars \\
                            &\mid  x = y & x,y \in \Vars \\
                            &\mid  x = y + 1 & x,y \in \Vars \\
                            &\mid  \last(x) & x,y \in \Vars \\
                            &\mid  \first(x) & x,y \in \Vars \\
    \end{align*}
    \caption{Syntax of Boolean Expressions}
\end{figure}

\begin{figure}[h]
    \begin{align*}
        \intro*\forString
        \defined & \,\, x & x \in \Vars \\
                 & \mid w \in \Sigma^* \\
    \end{align*}
    \caption{Syntax of String Expressions}
\end{figure}

\subsection{Denotational Semantics of Variable-Free For Programs}

\subsection{Simplified Syntax}

\end{document}
